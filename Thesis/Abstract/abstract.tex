% ************************** Thesis Abstract *****************************
% Use `abstract' as an option in the document class to print only the titlepage and the abstract.
\begin{abstract}

As per the definition of socio-technical systems, the role of 'people' in software projects is a well-known factor for their long-term, sustainable success. The size and the stability of a software development team are the often cited attributes in the 'people' category as being of greatest significance to stakeholders. Team size is usually considered as a measure of the number of developers that modify a project source code; while team stability is typically the cumulative time that each team member has worked with their fellow team members.

The impact of inappropriately sized or less stable teams can have profound effects on budget and time-scales: given that up to 80\% of development budgets are consumed in the maintenance phase of the lifecycle, these team-building factors are among the key aspects in a software project. While existing research establishes correlations between the size and stability of the development team and software fault-proneness, there is limited research investigating the impact on maintainability. Therefore, it is essential to gain an insight into how these aspects of team composition influence and are influenced by the internal characteristics of a software project, that inevitably drive the maintenance costs of software. 

Within this research the GoogleCode open-source forge is mined and, using the Chidamber and Kemerer design metrics suite, the impact of development team size and stability on the internal structural attributes of software is isolated and quantified. Drawing on prior research correlating these internal attributes with external attributes, the impact on maintainability is deduced and insight is given to the structural aspects that drive the increased fault-proneness that previous research has associated with larger, less stable development teams. 

This research finds that those structural attributes that have been established to correlate to fault-proneness - coupling, cohesion, modularity - show degradation as team sizes increase or team stability decreases. That degradation in the internal attributes of the software is associated with a deterioration in the sub-attributes of maintainability; changeability, understandability, testability and stability.
\end{abstract}
