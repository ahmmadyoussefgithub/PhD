% ************************** Thesis Abstract *****************************
% Use `abstract' as an option in the document class to print only the titlepage and the abstract.
\begin{abstract}
Software development is intrinsically a human activity and the role of the development team has been established as the among the most decisive of all project success factors. Prior research has proven empirically that team size and stability are linked to stakeholder statisfaction, team productivity and fault-proneness. There is, however, limited research investigating the impact of these factors on software maintainability - a crucial aspect given that up to 80\% of development budgets are consumed in the maintenance phase of the lifecycle. 

 This research sheds light on how these aspects of team composition influence the structural attributes of the developed software that, in turn, inevitably drive the maintenance costs of software. This thesis asserts that new and broader insights can be gained by measuring these internal attributes of the software rather than the more traditional approach of measuring its external attributes, also enabling practitioners to measure and monitor key indicators throughout the development lifecycle taking remedial action where appropriate.

Within this research the GoogleCode open-source forge is mined and a sample of 1,674 Java projects are selected for further study. Using the Chidamber and Kemerer design metrics suite, the impact of development team size and stability on the internal structural attributes of software is isolated and quantified. Drawing on prior research correlating these internal attributes with external attributes, the impact on maintainability is deduced. 

This research finds that those structural attributes that have been established to correlate to fault-proneness - coupling, cohesion, modularity - show degradation as team sizes increase or team stability decreases. That degradation in the internal attributes of the software is associated with a deterioration in the sub-attributes of maintainability; changeability, understandability, testability and stability.
\end{abstract}
