%!TEX root = ../thesis.tex

\chapter*{Glossary of Terms} 
\addcontentsline{toc}{chapter}{Glossary of Terms}

Note that the definition of the terms below is offered in the context in which they are mentioned within this thesis.

\textbf{Bonferroni correction}: An adjustment made to p-values when several statistical hypothesis tests are simultaneously performed on a single data set. \newline
\textbf{Branch}: A duplication of a source code directory structure within a VCS. \newline
\textbf{Class}: The template that defines the behaviour and/or state of objects of its type. \newline
\textbf{Code clone}: Sequences of duplicated source control residing across multiple VCS repositories. \newline
\textbf{Collinearity}: Correlation between two or more independent variables in a regression model. \newline
\textbf{Commit}: A set of  changes made to the source code in a software repository. \newline
\textbf{Committer}: An contributer who modifies source code in a software repository. \newline
\textbf{Confounding factors}: Confounding factors are those that influence both the dependent and independent variables
within a model causing an association to be made which may not be genuine. \newline
\textbf{Coupling}: The degree to which components within software systems are interdependent. \newline
\textbf{Cohesion}: The measure of the extent to which functionality within a single component belongs together. \newline
\textbf{Functional Complexity}: Functional complexity which has no single definition but generally refers to the degree of sophistication in the logic encoded within a software system. \newline
\textbf{Database schema}: The definition of the structure of the database, including its tables and the relationships between them. \newline
\textbf{Dependent variable}: The variable that is subject to testing and measurement in an experiment . \newline
\textbf{External attributes}:  These are externally visible properties which manifest in how the software relates to its environment. Examples include maintainabilty and fault-proneness. \newline
\textbf{Fault-proneness}: The extent to which software exhibits 'faults' which are structural imperfections which causes a system not to perform its required function. \newline
\textbf{FLOSS}: Free/Libre Open Source Software (FLOSS) is developed by informal collaborative networks of programmers. Source code is openly shared to encourage others to build upon the software. \newline
\textbf{Forge}: A collaborative platform designed to facilitate the creation of a community of developers to collaborate on the creation of software. Offers software development and management tools. \newline
\textbf{Fork}: The process of creating an alternate and independent software development stream from an existing project. \newline
\textbf{Integrated Development Environment (IDE)}: An application that provides comprehensive features to  support programmers in software development. \newline
\textbf{Independent variable}: The variable that is controlled in an  experiment to measure the effects on the dependent variable. \newline
\textbf{Inheritance}: The hierarchical arrangement of classes such that a child class derives behaviour from its parent. \newline
\textbf{Internal attributes}: These can be measured through direct observation of the software artefacts. Examples include structural properties such as coupling and cohesion. \newline
\textbf{Linear mixed models}: A form of linear regression that allows for both �fixed effects� that apply to all groups and �random effects� that apply individually to subgroups within data sets. \newline
\textbf{Linear regression}: A statistical model that attempts to establish a linear relationship between dependent and independent variables. \newline
\textbf{Maintainability}: The ease with which a software system or component can be modified to correct faults, improve performance or other attributes, or adapt to a changed environment. \newline
\textbf{Mann-Whitney U test}: Used to test the null hypothesis that two samples come from the same population or whether observations in one sample tend to be larger than observations in the other.
\textbf{Method}: A programmed procedure defined in a class that is included in all its instances. \newline
\textbf{Modularity}: The extent to which a system�s functionality is logically partitioned into independent components. \newline
\textbf{Module}: A logical grouping of related components making up part of a software application. \newline
\textbf{Multivariate}: A statistical model using multiple variables to predict an outcome. \newline
\textbf{Normal distribution}: A probability distribution symmetric about the mean. Observations are more frequent near the mean. When plotted, appears as a bell curve. \newline
\textbf{Object}: An instance of a class. \newline
\textbf{Object Oriented}:  A programming paradigm based around 'objects' rather than 'actions'. \newline
\textbf{Outlier}: An observation that is abnormally distant from other observations in a sample. \newline
\textbf{Principal Component Analysis}: A technique to reduce dimensions in a data set to transform it to a number of linearly uncorrelated dimensions while retaining the maximum variance within the data set. \newline
\textbf{Probability distribution}: A mathematical function providing probabilities of occurrence of different possible outcomes that a variable can assume. \newline
\textbf{p-value}: The probability of finding the observed, or more extreme, results given a true null hypothesis. \newline
\textbf{Regression coefficients}: In a linear model these are estimates of multipliers on independent variables. \newline
\textbf{Repository}: A structured data store archiving files, their revision histories and other associated meta-data. Used to facilitate and manage change to source code. \newline
\textbf{Revision}: A distinct changeset in a VCS repository. \newline
\textbf{Root Mean Square Estimate}: A measure of the distance between observations and the values predicted by a regression model. \newline
\textbf{R-squared}: A measure of how close the observations are to a regression line. \newline
\textbf{Scrum}: A framework for managing agile development. \newline
\textbf{Software artefact}: Tangible products of a software development process such as source code, compiled binary files or documentation. \newline
\textbf{Spearman correlation}:  A nonparametric measure of rank correlation rank-order assessing the relationship between two variables. \newline
\textbf{Stakeholders}: People or groups affected by the outcome of a software development process. \newline
\textbf{Standard error}: A  measure of the typical distance between data points and a regression line. \newline
\textbf{Static code analysis}: An analysis of software through direct inspection of its artefacts (particularly source code) without the execution of the software. \newline
\textbf{Structural complexity}: The measure of the degree of interactions between components in a software system. \newline
\textbf{T-statistic}: The ratio of the distance of the estimated value of a parameter from its regression line value to its standard error. \newline
\textbf{Univariate}: A statistical model using a single variable to predict an outcome. \newline
\textbf{Variance}: The expectation of the squared deviation of a random variable from its mean. \newline
\textbf{Version Control System}: A system that enables and tracks changes to a file or set of files to enable recovery to previous revisions. \newline

\end{itemize}